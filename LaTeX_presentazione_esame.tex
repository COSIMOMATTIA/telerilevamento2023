\documentclass[10pt,aspectratio=169]{beamer}
\usepackage{poly}

\title{Analisi dell'incendo avvenuto nel 2017 all'interno del parco Nazionale del Vesuvio}
%\subtitle{Using \LaTeX\ to prepare slides}
\author{Cosimo Mattia Scarpato}
\date{16/06/23}

\begin{document}

\maketitle

\section{Introduzione}

\begin{frame}{Telerilevamento Geo-ecologico}{\thesection \, \secname}
\begin{itemize}
\item In quest'elaborato si analizzano i danni subiti dalla vegetazione del parco Nazionale del Vesuvio a causa di un incendio doloso avvenuto nel 2017, confrontando lo stato di salute della vegetazione prima e dopo il disastro 
\end{itemize}
\end{frame}





\section{Metodi ed Analisi}

\begin{frame}[fragile]{Telerilevamento Geo-ecologico}
Metodi utilizzati:
\begin{itemize}
\item Download data set Copernicus
\item Elaborazione in Qgis 
\item Software per il telerilevamento R
\item Visuallizzazione delle immagini in RGB colori naturali e falsi colori 
\item Immagini satellitari Sentinel 2 
\item Indice di vegetazione NDVI, indice di bruciatura NBR


\end{itemize}
\end{frame}

\begin{frame}[fragile]{Telerilevamento Geo-ecologico}

\begin{figure}[h]
	\centering
	\includegraphics[width=0.5\textwidth]{vesuvio16.JPG}
	\caption{Prima visualizzazione dell'area in colori naturali 
 RGB 4, 3, 2 (2016)}
	\label{fig1}
\end{figure}
\end{frame}
\begin{frame}[fragile]{Telerilevamento Geo-ecologico}

\begin{figure}[h]
	\centering
	\includegraphics[width=0.5\textwidth]{vesuvio17.JPG}
	\caption{Prima visualizzazione dell'area in colori naturali 
 RGB 4, 3, 2 (2017)}
	\label{fig1}
\end{figure}
\end{frame}
\begin{frame}[fragile]{Telerilevamento Geo-ecologico}

\begin{figure}[h]
	\centering
	\includegraphics[width=0.5\textwidth]{vesuvio23.JPG}
	\caption{Prima visualizzazione dell'area in colori naturali 
 RGB 4, 3, 2 (2023)}
	\label{fig1}
\end{figure}
\end{frame}

\begin{frame}[fragile]{Telerilevamento Geo-ecologico}

\begin{figure}[h]
	\centering
	\includegraphics[width=0.5\textwidth]{vesuvio2016.JPG}
	\caption{Visualizzazione dell'area in falsi colori  
 RGB 8, 4, 3 (2016)}
	\label{fig1}
\end{figure}
\end{frame}
\begin{frame}[fragile]{Telerilevamento Geo-ecologico}

\begin{figure}[h]
	\centering
	\includegraphics[width=0.5\textwidth]{vesuvio2017.JPG}
	\caption{Visualizzazione dell'area in falsi colori  
 RGB 8, 4, 3 (2017)}
	\label{fig1}
\end{figure}
\end{frame}
\begin{frame}[fragile]{Telerilevamento Geo-ecologico}

\begin{figure}[h]
	\centering
	\includegraphics[width=0.5\textwidth]{vesuvio2023.JPG}
	\caption{Visualizzazione dell'area in falsi colori  
 RGB 8, 4, 3 (2023)}
	\label{fig1}
\end{figure}
\end{frame}

\begin{frame}[fragile]{Telerilevamento Geo-ecologico}

\begin{figure}[h]
	\centering
	\includegraphics[width=0.8\textwidth]{DVI.jpg}
	\caption{Confronto tra i DVI dei tre anni diversi, per mettere in risalto la quantità di vegetazione}
	\label{fig1}
\end{figure}
\end{frame}

\begin{frame}[fragile]{Telerilevamento Geo-ecologico}

\begin{figure}[h]
	\centering
	\includegraphics[width=0.8\textwidth]{NDVI.jpg}
	\caption{Confronto tra gli NDVI dei tre anni diversi, per quantificare lo stato di salute della vegetazione}
	\label{fig1}
\end{figure}
\end{frame} 

\begin{frame}[fragile]{Telerilevamento Geo-ecologico}

\begin{figure}[h]
	\centering
	\includegraphics[width=0.8\textwidth]{DVI1617.jpg}
	\caption{Differenza dei valori del DVI tra gli anni 2016 e 2017}
	\label{fig1}
\end{figure}
\end{frame} 

\begin{frame}[fragile]{Telerilevamento Geo-ecologico}

\begin{figure}[h]
	\centering
	\includegraphics[width=0.8\textwidth]{DVI2317.jpg}
	\caption{Differenza dei valori del DVI tra gli anni 2023 e 2017}
	\label{fig1}
\end{figure}
\end{frame} 

\begin{frame}[fragile]{Telerilevamento Geo-ecologico}

\begin{figure}[h]
	\centering
	\includegraphics[width=0.4\textwidth]{nbr2016.jpg}
	\caption{Indice di bruciatura nell'anno 2016 (pre incendio)}
	\label{fig1}
\end{figure}
\end{frame} 

\begin{frame}[fragile]{Telerilevamento Geo-ecologico}

\begin{figure}[h]
	\centering
	\includegraphics[width=0.4\textwidth]{nbr2017.jpg}
	\caption{Indice di bruciatura nell'anno 2017 (post incendio)}
	\label{fig1}
\end{figure}
\end{frame} 


\section{Conclusioni}

\begin{frame}
\frametitle{Telerilevamento Geo-ecologico}
In conclusione, le analisi effettuate hanno permesso di determinare che:
\begin{itemize}
    \item La zona che ha subito il maggior numero di danni è quella a sud del cratere 
    \item Dopo sei anni la vegetazione è quasi del tutto ricreciuta 
\end{itemize}

\end{frame}






%\backmatter

\end{document}
